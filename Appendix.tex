\appendix

\chapter{Matlab Code for network kernels}
\label{chapter:Appendix1}

\subsection{The Client ttkernel}

\begin{lstlisting}[language=Matlab, caption={client\textunderscore init.m}, label={listing:client_init}]
function client_init

% Distributed control system: server node
%
% Actuates controls sent from controller.

% Initialize TrueTime kernel
ttInitKernel('prioFP');   

% Client task
starttime = 0.0;
period = 0.010;
ttCreateTask('client_task', period, period, 'client_code');
ttAttachNetworkHandler('client_task')

\end{lstlisting}


\begin{lstlisting}[language=Matlab, caption={client\textunderscore code.m}, label={listing:client_code}]
function [exectime, data] = client_code(seg, data)

persistent y_3

switch seg
    case 1
        y_3 = ttGetMsg;         % Obtain sensor value, y is output, u is input
        if isempty(y_3)
            disp('Error in client: no message with actual measurement received!');
            y_3 = 0.0;
        else
            disp('Client success');
        end
        exectime = 0.0005;
    case 2
        ttAnalogOut(1, y_3);
        exectime = 0.0005;
    case 3
        exectime = -1; % finished
end
\end{lstlisting}

\subsection{The interference ttkernel}

\begin{lstlisting}[language=Matlab, caption={interference\textunderscore init.m}, label={listing:interference_init}]
function interference_init(arg)

% Distributed control system: interference node
%
% Generates disturbing network traffic with bandwidth arg (given in block mask)

% Initialize TrueTime kernel
ttInitKernel('prioFP');  % fixed priority scheduling

% Interference task
period = 0.01; 
offset = 0.0005;
data = arg;
ttCreatePeriodicTask('interference_task', offset, period, 'interference_code', data);

\end{lstlisting}

\begin{lstlisting}[language=Matlab, caption={interference\textunderscore code.m}, label={listing:interference_code}]
function [exectime, data] = interference_code(seg, data)

BWshare = data;
ran = ttAnalogIn(1);
if ran < BWshare
  ttSendMsg(3, 1, 256);   % send 80 bits to myself
end

while ~isempty(ttGetMsg) % read old received messages (if any)
end

exectime = -1;

\end{lstlisting}

