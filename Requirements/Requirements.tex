\label{chapter:final_prob_statement}
\noindent This chapter leads to the formulation of the final problem statement. It also states the requirement specifications the final solution of this project should fulfil in order to guide the design and development of the theory and implementation chapters. \\

\noindent The requirement specifications are formulated and based on the information exposed in the previous chapters.

\section{Requirement Specifications}

The requirements specifications are the delimiters of the final solution. Through their formulation, it should focus the design and development of the solution towards the fulfilment of the requirements. \\


\noindent After the solution has been designed and implemented, the testing results are verified against the requirements for benchmarking. For this reason, when formulated the requirements need to be unambiguous, verifiable, clear and feasible.\\


\noindent \textbf{Requirement specifications for the IMU board}

\begin{enumerate}
    \item The IMU board is able to measure high angular rates to allow precise hexacopter motions. 
    \item While grounded, IMU is calibrated at spirit level leading to a vertical take-off of the hexcopter when the transmitter's left stick is moved upwards. 
\end{enumerate}


\noindent \textbf{Requirement specifications for the Flight Controller (FC)}

\begin{enumerate}
    \item The FC update frequency allows the drone to react to the transmitter's control signal in less than a second. 
    \item ESCs are updated by the FC at 250Hz independent of the selected flight mode.
    \item Transmitter control signals are executed by the hexacopter's FC 250 times per second.
    \item Serial communication signals are executed by the hexacopter's FC 250 times per second.
    \item When in Acro flight mode, the hexacopter keeps rolling to the right/left if the transmitter's right stick is maintained right/left or even released. 
    \item When in Acro flight mode, the hexacopter keeps pitching up/down if the transmitter's right stick is maintained up/down or even released. 
    \item When in Acro flight mode, the hexacopter keeps yawing right/left if the transmitter's left stick is maintained right/left or even released. 
    \item When in Stabilise flight mode, the hexacopter auto-levels at IMU's spirit level from rolling if the transmitter's right stick is released. 
    \item When in Stabilise flight mode, the hexacopter auto-levels at IMU's spirit level from pitching if the transmitter's right stick is released. 
    \item When flipping the SF switch to Stabilise/Altitude Hold flight mode, the FC is not affected.
    \item When in Altitude Hold flight mode, the hexacopter maintains altitude with a deviation of 260mm (drone's height).
\end{enumerate}

\noindent \textbf{Requirement specifications for the Transmitter}

\begin{enumerate}
    \item The transmitter control signals does not interrupt the FC loop.
    \item When moving the right stick left/right the hexacopter rolls left/right.
    \item When moving the right stick up/down the hexacopter moves forward/backward.
    \item While mid-flight, moving the left stick up/down make the hexacopter ascend/descend.
    \item When moving the left stick left/right the hexacopter yaws left/right.
\end{enumerate}

\noindent \textbf{Requirement specifications for the Serial communication}

\begin{enumerate}
    \item The serial communication does not interrupt the FC loop.
    \item The serial communication samples altitude sensor data at 50Hz. 
\end{enumerate}

\section{Problem Statement}
This project comes as a continuation to few past projects that involved the development of a vision-based obstacle avoidance control algorithm\cite{p4project} ($3^{rd}$ semester) on a micro-quadrotor, and the development of a trajectory tracking controller for a fixed-wing\cite{p5project}. Both UA had off-the-shelf hardware (HW) flight controllers (FC) i.e. Pixhawk with open-source software (SW) flight controllers, i.e ArduPilot. Since these were ready-to-fly solutions, the focus was to build on top of the existing controller and make the UA behave as desired. It was one more abstraction layer added to the hardware.\\

\noindent Attempts were made in the beginning of this project to use an off-the-shelf HW FC with a light-weight SW FC meaning a SW FC with fewer layers of abstraction to the HW. Most HW FC on the market are designed for specific SW FC and require extensive work to fit them to other SW FC. Appendix \ref{appendix:3} shortly describes the SW setup required for an off-the-shelf HW FC such as the Pixhawk2 as attempted at the beginning of this project.\\

\noindent It was decided to use an open-source hardware such as Arduino Uno to build the HW FC and to upload a modified SW FC suitable for Arduino Uno to fly a hexacopter in Stabilise and Altitude Hold mode. The SW FC is based on \cite{website:ymfc} - an auto-level SW FC to fly a quadrotor.\\

\noindent Having exposed information about the hardware and software setup and requirements for the developed solution, the problem statement for this project is stated below with clearly defined goals.\\ 

\noindent \textbf{In this project it is sought to develop and implement an open-source hardware and software controller for stable hovering in regards to the hexacopter's attitude and altitude. The HW and SW implementation is done on Arduino Uno. The controller entails that the user is to start and fly the drone switching to the semi-autonomous for altitude hold and then switch back to full user control for landing.}\\

\noindent The hovering capability refers to the altitude control of the UA and is constructed as an semi-autonomous flight mode, meaning that the altitude is mantained without user input.\\

\noindent Moreover, the flight of UA is done in stabilise flight mode, which means that although the user is controlling the UA, the drone can auto-level roll and pitch when the user releases the transmitter sticks.\\

\noindent Taking-off and landing is user-controlled through a remote controller.

