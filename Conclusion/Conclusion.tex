\subsection{Discussion}
The paper aimed to show the effects of network delay and packet loss on the positional controller of a centralized swarm of drones. The results obtained from the simulations and experiments suggest that the simulation has a more stable controller to the network effects however it is needed to consider that although the simulation tried to resemble the experiments setup not all conditions were met, the most important aspect being network interference which was the main factor lowering the network LQ to an average of 0.6.  Other factors include the network model used by the Crazyflie Radio which is time-divion multiplexing (TDM) while the simulations used a wireless interface. More investigations need to be done for the simulation to resemble the real network environment, but from the results it can be concluded that controllers are destabilised by network effects about 10 times faster in real-time than in simulations. 

\subsection{Conclusion}
The paper presents both the non-linear and linear model of the Crazyflie. The classical control theory was used for developing the position and attitude controllers of the drone. These were simulated in Simulink and included in a network loop using the Matlab toolbox TrueTime. Developing the position controller on the x-axis revealed a stable but a slow converging system for a hovering drone. Further developments are needed to improve the response time of the system.\\ 

\noindent For running the experiments a swarm framework was needed in order to fly the drones autonomously. A ROS framework for controlling in real-time a swarm of drones was created. The radio link quality was observed in drones' hover mode.

Having simulated network delay and packet loss on a distributed real-time controller generating more than 2000 samples, a stability curve for the position controller on the x-axis has been drawn. The controller remains stable up to a 0.23 seconds delay and 0.03 probability of packet loss. These results were close to the experiments where drones appeared stable up to 0.25 s with an average packet loss of 0.3.

Performing the experiments in hover mode the link quality of the radio was affected by the data rate and the logging of the drone (ACK and other data). The experiments investigated the network effects on the z-axis controller. However, even though the simulation was analysing the x-axis controller the delay that would render the control unstable is very close to the delay found through experiments.