% \section{WS2: Preparation of Components}

% The 2 Crazyflie had to be checked for damages caused by previous uses such as bent rotors, defect wiring, etc. Failures to check for these issues take a long time to be found. Spare parts and batteries are crucial when working with nano-quadrotors (nano-drones will also be used throughout the report). Having few extra batteries charged ensures continuity of work.\\

% Using the Crazyflies with the Optitrack means using reflective markers on a platform that is already small. Finding a good position for the reflective markers means:
% \begin{itemize}
%     \item the rotors do not hit them;
%     \item they do not form an equilateral form such as an isosceles triangle, square;
%     \item the minimum required markers: 3, recommended: 4;
%     \item the Optitrack software sees all markers continuously.
%     \item the markers and glue do not overthrow the center of mass (COM);
% \end{itemize}

% The final position of the reflective markers can be seen in Figure \ref{}.Description for pic: first drone has 3 markers and is the leader of the swarm and the second drone has 4 markers to be distinguishable from the master.\\


% \textcolor{red}{Did it influence this with anything?}\\

% After the reflective markers have been placed, the drones have to be identified by the Optitrack software, MovieTracker 2.0.0, as rigid bodies. The Crazyflies have been added as rigid bodies with a name that can be recognized by the client: $cf1$ as master and $cf2$ as follower. Their representation in the tracker can be seen in Figure \ref{}.

