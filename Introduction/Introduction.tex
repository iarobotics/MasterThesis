\label{Introduction}

The purpose of this project refers to controlling systems that communicate over a network. It aims to investigate and implement a control strategy for systems communicating over a network with delay and information loss. Emphasis is placed in modelling of the system and state estimation. Signal measurements is also of importance.\\
Specifically, this paper deals with a swarm of nano-quadrotors. It aims to implement a centralized control strategy for autonomously flying the quadrotors inside the WNS Lab at AAU given a trajectory. Centralization is needed due to the performance restrictions imposed by the nano-quadrotors.\\


The context of controlling several units distributed over a network is widely seen in industry for a variety of system, not only drones. Specifically for drones, the domains employing configurations of drone warms is increasing but currently not limited to multiple-perspective photography, mapping, joint-task solving, military counter-strategies \cite{web_swarm}, surveillance, first-responders for emergencies, search-and-rescue \cite{web_coop}. \\


In this project the nano-quadrotor is at first mathematically modelled in order to develop an attitude and position controller. Signal measurements are simulated using Simulink and tested in the laboratory. The results and conclusions are specified through comparison between simulations and real-time testing. The facilities for real-time testing are given by the laboratory: 2 nano-quadrotors Crazyflie2.0, 1 Crazyflie Radio, Optitrack motion-capture system, Smart-City environment. The control strategy for the 2 drones implies one drone to act as the master of the swarm while the rest have to follow. A representation of the system is depicted in Figure \ref{}.\\


Besides the hardware above mentioned, the software entails the creation of 1 client per drone connecting to the server which hosts the Optitrack system. The clients are based on socket programming to receive position data from the server. Using the equations of motion obtained from modelling, the client communicates velocity commands to the drones in order to reach a specific position part of a trajectory. Furthermore, modelling and simulation is done through the use of Matlab and Simulink.

\subsection{Literature Review}

Quadrotors know today a growth in the industry due to their ease of fabrication, modelling, simulation and research \cite{modellingcrazy}. The possibilities for testing using rotor drones compared to i.e. fixed-wings has more advantages for researching and developing such as indoor testing in enclosed spaces, rapid response and manoeuvrability. As there are a multitude of drone configurations so are there a multitude of models and control strategies for manual, semi-autonomous and fully autonomous such as \cite{modellingcrazy}, \cite{beard_quadrotor}, \cite{beard2012small} and \cite{corke2017robotics} to name just a few. However when it comes to control of a drone swarm the literature is still growing and the applications limited. The same cannot be said about robotic swarms which usually implies ground robots. In 2015, \cite{swarm_1} has successfully flown 6 Crazyflie drones using 3 radio antennas. In 2017 \cite{preiss2017crazyswarm} developed further the system and reached an impressive swarm of 49 Crazyflie drones controlled with 3 antennas. The novel idea comes from the state estimation control as 4 markers for the optical motion sensing system cannot be placed in 49 different configuration on a nano-quadrotor. While \cite{preiss2017crazyswarm} regards centralised coordination and communication, other projects such as \cite{chaumette2011carus} investigate decentralised behaviour of each drone affecting the swarm behaviour through inter-communication. Therefore the type of network connecting to or in-between drones is very important as \cite{yanmaz2018drone} mentions. This project hence investigates the effects of network delay and packet loss on a drone swarm control. The system is centralised through a ground station that runs the position controller off-board and communicates over a wireless network motor velocity commands for waypoint following. It uses ROS to interface between the optical motion sensor Optitrack and the ground station. ROS is used as a framework for interfacing either one or multiple drones as in \cite{HoenigMixedReality2015} and \cite{educationalresearch}.